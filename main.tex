\documentclass[12pt,reqno]{amsart}
\usepackage{amsmath, amssymb}
\usepackage[margin=1in]{geometry}

\begin{document}
    \title{Subsets of $\mathbb{F}_p^d$ where no three elements sum to zero.}
    \author{Ave Robertson}    
    \begin{abstract}
        In this paper, we do some shit that I havent quite decided yet.
    \end{abstract}
    \maketitle

    \section{Introduction and Background}
    
    Deep mathematical richness often arises from the simplest of structures, and the card game \textit{SET} exemplifies this. Each SET card contains four features, with three possibilities for each feature. These are:
    \begin{itemize}
        \item Colour (red, green or purple),
        \item Shading (solid, striped, open),
        \item Shape (diamond, squiggle, oval),
        \item Number of Shapes (one, two, or three).
    \end{itemize}

    Three cards are said to form a SET if, for each of the four features, the three cards have that feature as either all the same or all different.
\end{document}