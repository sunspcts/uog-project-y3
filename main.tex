\documentclass[12pt,reqno]{amsart}
\usepackage{amsmath, amssymb}
\usepackage[margin=1in]{geometry}
\usepackage{graphicx}
\usepackage{subfig}

\begin{document}
    \title{Subsets of $\mathbb{F}_p^d$ where no three elements sum to zero.}
    \author{}    
    \begin{abstract}
        In this paper, we do some stuff that I havent quite decided yet.
    \end{abstract}
    \maketitle

    \section{Introduction and Background}
    
    Deep mathematical richness often arises from the simplest of structures, and the card game \textit{SET} exemplifies this. Each SET card contains four features, with three possibilities for each feature. These are:
    \begin{itemize}
        \item Colour (red, green or purple),
        \item Shading (solid, striped, open),
        \item Shape (diamond, squiggle, oval),
        \item Number of Shapes (one, two, or three).
    \end{itemize}

    Three cards are said to form a SET if, for each of the four features, the three cards have that feature as either all the same or all different.

    \begin{figure}[h]%
        \centering
        \subfloat[\centering Colour, Shading, Shape and Number are all different.]{{\includegraphics[width=0.4\linewidth]{GOODSET}} }
        \qquad
        \subfloat[\centering The second and third card have the same colour, so colour is \textit{not} all the same or all different for the three cards.]{{\includegraphics[width=0.4\linewidth]{BADSET}} }

        \caption{An example of a valid SET and an invalid SET}\label{fig:GOODVSBAD}
    \end{figure}

    In the figure above, by these rules (A) is a valid SET, while (B) is not. Note that B \textit{would} be a valid set if the diamond was blue.
\end{document}  